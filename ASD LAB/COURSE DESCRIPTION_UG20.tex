\documentclass[11pt]{exam}
\usepackage[a4paper,width=170mm,top=20mm,bottom=25mm]{geometry}
%\usepackage[top=1in, bottom=1.25in, left=1.25in, right=1.25in]{geometry}
\usepackage{graphicx,lastpage}
\usepackage{upgreek}
\usepackage{censor}
\usepackage{tabularx}
\usepackage{xcolor}
\usepackage{amsmath}
%\usepackage{cleveref}
%\usepackage{tabularx,pbox}
%\usepackage[nopar]{lipsum}
\usepackage{longtable}
\usepackage{multirow}
\usepackage[inline]{enumitem}
\usepackage{float}
%\usepackage{tikz}
\usepackage{amssymb}
\usepackage{pdfrender}
\usepackage{pgfplots}  
\pgfplotsset{width=10cm,compat=1.8} 
\newcommand*{\boldcheckmark}{%
	\textpdfrender{
		TextRenderingMode=FillStroke,
		LineWidth=.5pt, % half of the line width is outside the normal glyph
	}{\checkmark}%
}
\setlength{\parindent}{0pt}
	\definecolor{blue(pigment)}{rgb}{0.2, 0.2, 0.6}
\flushbottom
\usepackage[normalem]{ulem}
\renewcommand{\thesection}{\large \Roman{section}}
\pagestyle{headandfoot}

\begin{document}

%==============================================================
	\begin{minipage}{\linewidth}
	\begin{minipage}{0.14\linewidth}%
		\includegraphics[width=0.85\textwidth]{iare.png}\end{minipage}
	\begin{minipage}[r]{0.86\textwidth}%
		\noindent
		\begin{center}	
			\renewcommand{\arraystretch}{1.0}
			\textcolor{blue}{\Large \bfseries INSTITUTE OF AERONAUTICAL ENGINEERING}\\
			%\hspace*{5.2cm} 
			\textcolor{blue}{\Large (Autonomous)} \\
			%\hspace*{4.7cm}
			\small Dundigal, Hyderabad - 500 043 \\  [3pt] 
			\large \bfseries AERONAUTICAL ENGINEERING \\\vspace{2pt}
	\textcolor{red}{\large \bfseries COURSE DESCRIPTION} \\\vspace{3pt}
\end{center}
\end{minipage}\end{minipage}
\par
\newcolumntype{C}[1]{>{\centering\arraybackslash}p{#1}}
\newcolumntype{R}[1]{>{\raggedright\arraybackslash}p{#1}}
\renewcommand{\arraystretch}{1.2}
%\vspace{0.5cm}
\begin{flushleft}
	\begin{tabular}{|>{\raggedright\arraybackslash}p{3.5cm}  | >{\centering\arraybackslash}p{2.2cm}  |   >{\centering\arraybackslash}p{2.2cm} | >{\centering\arraybackslash}p{2.2cm}|
	>{\centering\arraybackslash}p{2.2cm}  | >{\centering\arraybackslash}p{3cm} |}
	\hline
		Course Title & \multicolumn{5}{l|} {\textbf{AEROSPACE STRUCTURAL DYNAMICS LABORATORY}} \\ \hline
		
	Course Code  & \multicolumn{5}{l|}{AAEC45}  \\ \hline
	
	Program & \multicolumn{5}{l|}{B.Tech}  \\ \hline
	
	Semester  & \multicolumn{5}{l|}{VII}  \\ \hline
	
	Course Type  & \multicolumn{5}{l|}{ Laboratory }  \\ \hline
	
	Regulation  & \multicolumn{5}{l|}{ UG-20}     \\ \hline
	
	\multirow{3}{*}{Course Structure} & \multicolumn{3}{c|}{Theory}      & \multicolumn{2}{c|}{Practical} \\ \cline{2-6} & \multicolumn{1}{c|}{Lecture} & \multicolumn{1}{c|}{Tutorials} & Credits & Laboratory      & Credits      \\ \cline{2-6} 
	& \multicolumn{1}{c|}{-}       & -  & -       & 3              & 1.5         \\ \hline
	Course Coordinator  &  \multicolumn{5}{l|}{Mr G Shiva krishna, Assistant Professor}     \\ \hline
\end{tabular}
\end{flushleft}
\vspace{-0.5cm}
\begin{flushleft}
	\textcolor{blue}{\section{\large \bfseries COURSE PRE-REQUISITES:}}
	\begin{tabular}{|>{\centering\arraybackslash}p{3cm}  | >{\centering\arraybackslash}p{4cm}  |   >{\centering\arraybackslash}p{3cm} |>{\centering\arraybackslash}p{5cm}|} 
		\hline 		
		\textbf{Level}&	\textbf{Course Code}&	\textbf{Semester}&	\textbf{Prerequisites}\\ 
		\hline
		B.Tech& AMEC24	&II	& Aircraft Stability and Control	\\ 
		\hline
\end{tabular}\end{flushleft}
\textcolor{blue}{\section{\large \bfseries COURSE OVERVIEW:}}
This course focuses on mechanical devices that are designed to have mobility to perform certain functions. In this
process they are subjected to some forces. This course will provide the knowledge on how to analyze the motions of
mechanisms and design mechanisms to give required strength. This includes relative static and dynamic force analysis
and consideration of gyroscopic effects on aero planes, ships, automobiles like two wheelers and four wheelers.
Balancing of rotating and reciprocating masses, friction effect in brakes clutches and dynamometers are also studied.
Mechanical vibrations give an insight into the various disturbances while designing vibratory systems.

\begin{flushleft}\vspace{-0.75cm}
\textcolor{blue}{\section{\large \bfseries MARKS DISTRIBUTION:}}
	\begin{tabular}{|>{\centering\arraybackslash}p{5cm}  | >{\centering\arraybackslash}p{3.75cm}  |   >{\centering\arraybackslash}p{3.75cm} |>{\centering\arraybackslash}p{2.8cm}|}
		\hline 
	\textbf{Subject}&	\textbf{SEE Examination}&	\textbf{CIE Examination}&	\textbf{Total Marks}\\ 
	\hline
AEROSPACE STRUCTURAL DYNAMICS LABORATORY	&	70 Marks&	30 Marks&	100\\ 
	\hline
	\end{tabular}
\end{flushleft}\vspace{-1cm}
\begin{flushleft}
\textcolor{blue}{\section{\large \bfseries DELIVERY / INSTRUCTIONAL METHODOLOGIES:}}
	\begin{tabular}{|>{\centering\arraybackslash}p{0.3cm}  | >{\centering\arraybackslash}p{2.3cm}  |   >{\centering\arraybackslash}p{0.5cm} |>{\centering\arraybackslash}p{2.4cm}|>{\centering\arraybackslash}p{0.5cm}  | >{\centering\arraybackslash}p{3cm}  |   >{\centering\arraybackslash}p{0.5cm} |>{\centering\arraybackslash}p{4cm}|}
	\hline 

\boldcheckmark &  Demo Video  & \boldcheckmark& Lab Worksheets &  \boldcheckmark & Viva Questions  & \boldcheckmark &  Probing further Questions \\ \hline
\end{tabular}
\end{flushleft}
\vspace{-2.5cm}
\newpage
\textcolor{blue}{\section{\large \bfseries EVALUATION METHODOLOGY:}}
Each laboratory will be evaluated for a total of 100 marks consisting of 30 marks forinternal assessment and 70 marks for semester end lab examination. Out of 30 marks ofinternal assessment, continuous lab assessment will be done for 20 marks for the day today performance and 10 marks for the final internal lab assessment. \\
\textbf{Semester End Examination (SEE):}The semester end labexamination for 70 marks shall be conducted by two examiners, one of them beingInternal Examiner and the other being External Examiner, both nominated by thePrincipal from the panel of experts recommended by Chairman, BOS.
The emphasis on the experiments is broadly based on the following criteria given in Table: 1
\begin{longtable}{|C{4cm}|C{5.8cm}|C{5.6cm}|}
	\hline
	&Experiment Based	&	Programming based\\ \hline
20 \%&	Objective&	Purpose\\ \hline
20 \%&	Analysis&	Algorithm\\ \hline
20 \%&	Design&	Programme\\ \hline
20 \%&	Conclusion&	Conclusion\\ \hline
20 \%&	Viva&	Viva\\ \hline
\end{longtable}
\textcolor{blue}{\textbf{\large  Continuous Internal Assessment (CIA):}}\\
CIA is conducted for a total of 30 marks (Table 1), with 20 marks for continuous lab assessment during day to day performance, 10 marks for final internal lab assessment. 
\begin{longtable}{|C{2.5cm}|C{4.3cm}|C{4.5cm}|C{3.5cm}|}
	\hline
\centering \textbf{Component}&	\multicolumn{2}{c|}{\textbf{Laboratory}}    &	 \multirow{2}{*}{\textbf{Total Marks}} \\ \cline{1-3}
\textbf{Type of Assessment}&	\textbf{Day to day performance}&	\textbf{Final internal lab assessment}&	\\\hline
CIA Marks&	20&	10&	30\\\hline
\end{longtable}
\textcolor{blue}{\textbf{\large Continuous Internal Examination (CIE):}}\\
One CIE exams shall be conducted at the end of the 16th week of the semester. The CIE exam is conducted for 10 marks of 3 hours duration.
\begin{enumerate}
	\item \textbf{Experiment Based }
\begin{longtable}{|C{2.2cm}|C{2.2cm}|C{2.2cm}|C{2.42cm}|C{2.42cm}|C{2.2cm}|}
	\hline
Objective&	Analysis&	Design&	Conclusion&	Viva&	Total\\\hline
2&	2&	2&	2&	2&	10\\\hline
\end{longtable}
\item \textbf{Programming Based }
\begin{longtable}{|C{2.2cm}|C{2.2cm}|C{2.2cm}|C{2.42cm}|C{2.42cm}|C{2.2cm}|}
	\hline
	Objective&	Analysis&	Design&	Conclusion&	Viva&	Total\\\hline
	-&	-&	-&	-&	-&	0\\\hline
\end{longtable}
\end{enumerate}\vspace{-1cm}
\newpage
\textcolor{blue}{\section{\large \bfseries COURSE OBJECTIVES:}}\vspace{-0.4cm}

%	\flushleft\textbf{\textcolor{blue}{\large COURSE OBJECTIVES:}}\\		
\textbf{The students will try to learn:}\vspace{-0.4cm}
\newcolumntype{C}[1]{>{\centering\arraybackslash}p{#1}}
\newcolumntype{R}[1]{>{\raggedright\arraybackslash}p{#1}}
\renewcommand{\arraystretch}{1.2}
\begin{flushleft}	
	\begin{longtable}{|C{1.5cm}|R{15cm}|}
		\hline
		I &The basic principles of kinematics and there lated terminology of machines. \tabularnewline
		\hline
		II & The Discriminate mobility; enumerate links and joints in the mechanisms. \tabularnewline
		\hline
		III & The concept of analysis and formulation of different mechanisms. \tabularnewline
		\hline
	 
		\hline	
	\end{longtable}
\end{flushleft}\vspace{-2cm}
\textcolor{blue}{\section{\large \bfseries COURSE OUTCOMES:}}\vspace{-0.4cm}
\textbf{After successful completion of the course, students should be able to:}
\renewcommand{\arraystretch}{1.1}\vspace{-0.4cm}
\begin{flushleft}
	\begin{longtable}{|C{1.2cm}|R{13cm}|C{2cm}|}
		\hline
	
		CO 1&\textbf{Choose} \textcolor{blue}{ the function of governors and gyroscopes in }\textcolor{red}{  aerospace systems. }   	& Understand\tabularnewline\hline
		CO 2&\textbf{ Perform } \textcolor{blue}{ static and dynamic force analysis of} \textcolor{red}{ mechanisms.
			the complex surfaces	 }	& Apply\tabularnewline
		\hline
			CO 3& \textbf{Calculate } \textcolor{blue}{the balancing forces and reciprocating masses} \textcolor{red}{in mechanical systems.}	&Apply \tabularnewline\hline
		CO 4& 
			\textbf { Analyze } \textcolor{blue}{longitudinal and lateral vibrations } \textcolor{red}{ in mechanical systems. }	& Apply\tabularnewline
		\hline
			CO 5& 	\textbf{Design and evaluate  } \textcolor{blue}{the critical speed of  } \textcolor{red}{ rotating shafts and mechanisms.}	& Apply \tabularnewline \hline
			CO 6& 	\textbf{ Investigate} \textcolor{blue}{free and forced vibrations } \textcolor{red}{in beam structures, particularly cantilever beams}	& Apply \tabularnewline 
		\hline
	\end{longtable}
\end{flushleft}\vspace{-1.5cm}
\textcolor{blue}{\section{\large \bfseries  PROGRAM OUTCOMES: }}\vspace{-0.4cm}
\begin{flushleft}
	\begin{longtable}{|>{\centering\arraybackslash}p{1.6cm}  | >{\raggedright\arraybackslash}p{15cm}  | }
		\hline
		\multicolumn{2}{|c|}{\textbf{Program Outcomes}}  \\ \hline
		\endhead
		\hline
		PO 1&	\textbf{Engineering knowledge:} Apply the knowledge of mathematics, science, engineering fundamentals, and an engineering specialization to the solution of complex engineering problems.\\ \hline
		PO 2&	\textbf{Problem analysis:} Identify, formulate, review research literature, and analyze complex engineering problems reaching substantiated conclusions using first principles of mathematics, natural sciences, and engineering sciences.\\ \hline
		PO 3&	\textbf{Design/Development of Solutions:} Design solutions for complex Engineering problems and design system components or processes that meet the specified needs with  appropriate consideration for the public health and safety,
		and the cultural, societal, and Environmental considerations\\ \hline
		PO 4&	\textbf{Conduct Investigations of Complex Problems:} Use research-based knowledge and research methods including design of experiments, analysis and interpretation of data, and synthesis of the information to provide valid conclusions.\\ \hline
		PO 5&	\textbf{Modern Tool Usage: }Create, select, and apply appropriate techniques, resources, and modern Engineering and IT tools including prediction and modelling to complex Engineering activities with an understanding of the limitations\\ \hline
		
		PO 6& \textbf{The  engineer  and  society:} Apply reasoning informed by the contextual knowledge to assess 
		societal,  health,  safety,  legal  and  cultural  issues  and  the  consequent  responsibilities  relevant  to 
		the professional engineering practice. \\ \hline
		PO 7&  \textbf{Environment  and  sustainability:}  Understand  the  impact  of  the  professional  engineering 
		solutions  in  societal  and  environmental  contexts,  and  demonstrate  the  knowledge  of,  and  need 
		for sustainable development. \\ \hline
		PO 8& \textbf{Ethics:}  Apply  ethical  principles  and  commit  to  professional  ethics  and  responsibilities  and 
		norms of the engineering practice. \\ \hline
		PO 9& \textbf{Individual and team work:} Function effectively as an individual, and as a member or leader 
		in diverse teams, and in multidisciplinary settings. \\ \hline
		PO 10&  \textbf{Communication: } Communicate  effectively  on  complex  engineering  activities  with  the 
		engineering  community  and  with  society  at  large,  such  as,  being  able  to  comprehend  and  write 
		effective  reports  and  design  documentation,  make  effective  presentations,  and  give  and  receive 
		clear instructions. \\ \hline
		PO 11&  \textbf{Project management  and  finance: } Demonstrate  knowledge  and  understanding  of  the 
		engineering  and  management  principles  and  apply  these  to  one’s  own  work,  as  a  member  and 
		leader in a team, to manage projects and in multidisciplinary environments. \\ \hline
		PO 12&	\textbf{Life-Long Learning:} Recognize the need for and having the preparation and ability to engage in independent and life-long learning in the broadest context of technological change\\ \hline
	\end{longtable}
\end{flushleft}\
\vspace{-2.0cm}
%\vspace{-0.5cm}
\textcolor{blue}{\section{\large \bfseries HOW PROGRAM OUTCOMES ARE ASSESSED:}}\vspace{-0.4cm}
\begin{flushleft}
	\begin{longtable}{|>{\centering\arraybackslash}p{1.6cm}  | >{\raggedright\arraybackslash}p{8.8cm}  |   >{\centering\arraybackslash}p{1.8cm} |>{\centering\arraybackslash}p{2.7cm}|}
		\hline
		\multicolumn{2}{|c|}{\textbf{Program}} & \textbf{Strength} & \textbf{Proficiency Assessed by} \\ \hline
	PO1 &Apply the knowledge of mathematics, science, engineering fundamentals, and an engineering specialization to the solution of complex engineering problems.	&2	&CIA,SEE	\\ \hline
	PO3 & Design solutions for complex Engineering problems and design system components or processes that meet the specified needs with  appropriate consideration for the public health and safety,	&3	&CIA,SEE	\\ \hline
	PO4 &Use research-based knowledge and research methods including design of experiments, analysis and interpretation of data, and synthesis of the information to provide valid conclusions.&3	&CIA,SEE	\\ \hline
	PO12 &Recognize the need for and having the preparation and ability to engage in independent and life-long learning in the broadest context of technological change.&2	&CIA,SEE	\\ \hline
	\multicolumn{4}{l}{\textbf{3 = High; 2 = Medium; 1 = Low}}\\ 
			\end{longtable}
	\end{flushleft}\vspace{-2.5cm}
	
\textcolor{blue}{\section{\large \bfseries HOW PROGRAM SPECIFIC OUTCOMES ARE ASSESSED:}}\vspace{-0.4cm}
\begin{flushleft}
	\begin{longtable}{|>{\centering\arraybackslash}p{1.8cm}  | >{\raggedright\arraybackslash}p{9cm}  |   >{\centering\arraybackslash}p{2cm} |>{\centering\arraybackslash}p{2cm}|}
		\hline
		\multicolumn{2}{|c|}{\textbf{Program}} & \textbf{Strength} & \textbf{Proficiency Assessed by} \\ \hline
		PSO 3 &		Make use of multi physics, computational fluid dynamics and flight simulation tools for building career paths towards innovative startups, employability and higher studies. &1	&CIA,SEE	\\ \hline
		\multicolumn{4}{l}{\textbf{3 = High; 2 = Medium; 1 = Low}}\\ 
		\end{longtable}
\end{flushleft}

\vspace{-1.5cm}
\textcolor{blue}{\section{\large \bfseries	MAPPING COURSE OUTCOMES LEADING TO THE ACHIEVEMENT OF PROGRAM OUTCOMES AND PROGRAM SPECIFIC OUTCOMES}}\vspace{-0.4cm}
\begin{flushleft}
	\begin{longtable}{|>{\centering\arraybackslash}p{4cm}  | >{\centering\arraybackslash}p{1.75cm}  |   >{\centering\arraybackslash}p{1.75cm} |>{\centering\arraybackslash}p{1.75cm}| >{\centering\arraybackslash}p{1.75cm}|>{\centering\arraybackslash}p{2.5cm}  | }
	\hline	
\multirow{2}{*}{\textbf{\begin{tabular}[c]{@{}c@{}}\small  COURSE \\\small OUTCOMES\end{tabular}}} & \multicolumn{4}{l|}{PROGRAM OUTCOMES} & PSO'S \\ \cline{2-5} 
	       &     PO 1    &   PO 3    &   PO 4    & PO12& PSO 3\\ \hline
CO 1	&   \boldcheckmark    &     &       &  &\\ \hline
CO 2	&      & \boldcheckmark      & \boldcheckmark      &  &\\ \hline
CO 3	&   \boldcheckmark  &      & \boldcheckmark      &  &\\ \hline
CO 4	&      &    \boldcheckmark   &      & \boldcheckmark &\\ \hline
CO 5	&  \boldcheckmark    &     &      &  &\boldcheckmark\\ \hline
CO 6	&      & \boldcheckmark     &       &  \boldcheckmark&\\ \hline

	\end{longtable}
\end{flushleft}\vspace{-1.75cm}
\textcolor{blue}{\section{\large \bfseries ASSESSMENT METHODOLOGY DIRECT:}}\vspace{-0.24cm}

\begin{flushleft}	
	\begin{tabular}{|>{\centering\arraybackslash}p{3cm}  | >{\centering\arraybackslash}p{2.2cm}  |   >{\centering\arraybackslash}p{3cm} |>{\centering\arraybackslash}p{2.5cm}|>{\centering\arraybackslash}p{2.6cm}  |>{\centering\arraybackslash}p{1cm}  | } 
		\hline 		
	CIE Exams            & \checkmark & SEE Exams       & \checkmark & Seminars               & -     \\ \hline
	Laboratory Practices &     \checkmark               & Student Viva    &       \checkmark          & Certification          & - \\ \hline

	\end{tabular}
\end{flushleft}
%\vspace{-1cm}
\textcolor{blue}{\section{\large \bfseries	ASSESSMENT METHODOLOGY INDIRECT:}}\vspace{-0.4cm}	
	\begin{longtable}{|C{1.2cm}|R{6cm}|C{1.3cm}|R{6.5cm}|}
		\hline
	\boldcheckmark&	Early Semester Feedback&	\boldcheckmark &	End Semester OBE Feedback\\\hline
	\textbf{X}&	\multicolumn{3}{l|}{Assessment of Mini Projects by Experts} \\\hline
\end{longtable}
\vspace{-1.0cm}
\textcolor{blue}{\section{\large \bfseries	SYLLABUS:}}	\vspace{-0.4cm}
  
	\centering
	\renewcommand{\arraystretch}{1.2}
\begin{longtable}{|>{\centering\arraybackslash}p{2.2cm}  | >{\raggedright\arraybackslash}p{13.8cm}  | }
		\hline 
		WEEK I & \textcolor{blue}{\textbf{Week-1: GOVERNORS}}\\
		\hline
		& To study the function of a Governor.\\
		\hline
		WEEK II & \textcolor{blue}{\textbf{ Week-2: GYROSCOPE}}\\
		\hline
		& To determine the Gyroscope couple.\\
		\hline
		WEEK III & \textcolor{blue}{\textbf{Week-3: STATIC FORCE ANALYSIS}}\\
		\hline
		&To draw free body diagram and determine forces under static condition.\\
		\hline
		WEEK IV & \textcolor{blue}{\textbf{Week-4: DYNAMIC FORCE ANALYSIS}}\\
		\hline
		&To draw free body diagram and determine forces under dynamic condition\\
		\hline
		WEEK V & \textcolor{blue}{\textbf{ Week-5: BALANCING}}\\
		\hline
		& To determine balancing forces and reciprocating masses.\\
		\hline
		WEEK VI & \textcolor{blue}{\textbf{Week-6: BEARINGS}}\\
		\hline
		&To determine the bearing life.\\
		\hline
		WEEK VII & \textcolor{blue}{\textbf{Week-7: LONGITUDINAL AND LATERAL VIBRATIONS}}\\
		\hline
		&To determine the longitudinal and transfer vibration\\
		\hline
		WEEK VIII & \textcolor{blue}{\textbf{Week-8: VIBRATION ANALYSIS OF SHAFT}}\\
		\hline
		&To determine critical speed of a shaft.\\
		\hline
		WEEK IX & \textcolor{blue}{\textbf{Week-09: MECHANISMS}}\\
		\hline
		&To design various mechanism and their inversions
		\\
		\hline
		WEEK X & \textcolor{blue}{\textbf{ Week-10: DIFFERENTIAL GEAR BOX}}\\
		\hline
		&To study automobile differential gear box\\
		\hline
		WEEK XI & \textcolor{blue}{\textbf{Week-11: FREE VIBRATION OF CANTIEVER BEAM}}\\
		\hline
		&T.To study Vibrations in beam Structures\\
		\hline
		WEEK XII & \textcolor{blue}{\textbf{Week-12: FORCED VIBRATION OF CANTIEVER BEAM}}\\
		\hline
		&To study Vibrations in beam Structures\\
		\hline
		
	\end{longtable}

\raggedright	\textcolor{blue}{\textbf{TEXTBOOKS}}\\
\begin{enumerate}
		\item  Joseph E. Shigley, “Theory of Machines and Mechanisms”, Oxford University Press, 4th Edition, 2010.
		\item Thomas Bevan, “Theory of Machines”, Pearson, 3rd Edition, 2009.
	\end{enumerate}
  \textcolor{blue}{\textbf{REFERENCE BOOKS:}}\\
\begin{enumerate}
	\vspace{-0.5cm}
	\item http://www.e-booksdirectory.com.

\end{enumerate}
\vspace{-1cm}
\textcolor{blue}{\section{\large \bfseries	COURSE PLAN:}}\vspace{-0.4cm}
The course plan is meant as a guideline. Probably there may be changes.

\begin{flushleft}
	\begin{longtable}{|>{\centering\arraybackslash}p{1cm}  | >{\raggedright\arraybackslash}p{10.5cm}  |   >{\centering\arraybackslash}p{1.5cm} |>{\centering\arraybackslash}p{1.8cm}|}
		\hline 
			\textbf{S.No}&	\centering{\textbf{ Topics to be covered}} &	\textbf{CO's}&\textbf{Reference}\\ 
		\hline
1&	 GOVERNORS &	Co1& book1	\\
\hline
2& GYROSCOPE	&		Co1& book1	\\
\hline
3&STATIC FORCE ANALYSIS	&		Co2& book1	\\
\hline
4&DYNAMIC FORCE ANALYSIS	&		Co2& book1	\\
\hline
5& BALANCING	&	Co3& book1	\\
\hline
6&BEARINGS&		Co3& book1\\
\hline
7& LONGITUDINAL AND LATERAL VIBRATIONS&		Co3& book1\\
\hline
8&VIBRATION ANALYSIS OF SHAFT&		Co4& book1	\\
\hline
9&MECHANISMS&			Co4& book2	\\
\hline
10& DIFFERENTIAL GEAR BOX&		Co5& book2	\\
\hline
11& FREE VIBRATION OF CANTIEVER BEAM &		Co5& book2	
\\
\hline
12& FORCED VIBRATION OF CANTIEVER BEAM&		Co1& book2	\\
\hline
	\end{longtable}
	\vspace{-2cm}
\end{flushleft}
\textcolor{blue}{\section{\large \bfseries	EXPERIMENTS FOR ENHANCED LEARNING (EEL):}}\vspace{-0.4cm}
	\begin{longtable}{|>{\centering\arraybackslash}p{1cm}  | >{\raggedright\arraybackslash}p{15cm}  |   }
	\hline
\textbf{S.No}&\textbf{\centering Design Oriented Experiments} \\
	\hline
	1&Design of a Gyroscopic Stabilizer
	\\
	\hline
	2& Analysis of Vibration Isolation in Aircraft Components\\
	\hline
	3& Dynamic Force Analysis of Landing Gear Mechanism	\\
	\hline
	4& Optimization of Balancing Techniques in Rotating Turbine Blades\\
	\hline
	5& Advanced Vibration Analysis of Multi-Mass Systems\\
	\hline
\end{longtable}
\vspace{2cm}
\flushleft \textbf{Signature of Course Coordinator}\hspace{8cm} \textbf{HOD,AE}\\\textbf{Mr G Shiva krishna.}\\


\end{document}
}