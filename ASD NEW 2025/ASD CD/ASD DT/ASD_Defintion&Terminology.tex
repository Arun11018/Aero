\documentclass[11pt,paper=a4,answers]{exam}
\usepackage{graphicx,lastpage}
\usepackage{upgreek}
\usepackage{censor}
\usepackage{tabularx}
\usepackage[dvipsnames]{xcolor}
\usepackage{amsmath}
%\usepackage{cleveref}
%\usepackage{tabularx,pbox}
\usepackage[nopar]{lipsum}
\usepackage{longtable}
\usepackage{multirow}
\usepackage[inline]{enumitem}
\usepackage{float}
\usepackage{tikz}
\usepackage{amssymb}
\usepackage{pdfrender}
\usepackage{booktabs}
\usepackage{dcolumn}
\usepackage{colortbl}
%\newcolumntype{C}{>{\centering\arraybackslash}p{3cm}}% for convenience you can also define a new column type

\renewcommand*{\arraystretch}{1.00}%... and increase the row height

\begin{document}
	\begin{minipage}{\linewidth}
	\begin{minipage}{0.14\linewidth}%
\includegraphics[width=0.85\textwidth]{iare.png}\end{minipage}
	%% \thispagestyle{empty}
	\begin{minipage}[r]{0.86\textwidth}%
		\noindent
		\begin{center}	
			\textcolor{blue}{\Large \bfseries INSTITUTE OF AERONAUTICAL ENGINEERING}\\
			%\hspace*{5.2cm} 
			\textcolor{blue}{\Large (Autonomous)} \\
			%\hspace*{4.7cm}
			\small Dundigal, Hyderabad - 500 043 \\  [3pt] 
			\vspace{5pt}
			\large \bfseries AERONAUTICAL ENGINEERING \\\vspace{5pt}
			%\framebox[1.1\width]{\small ACADEMIC YEAR 2021-2022} \par \vspace{5pt}
			\textcolor{red}{\large \bfseries DEFINITION AND TERMINOLGY} \\\vspace{3pt}
		\end{center}
	\end{minipage}
	\end{minipage}
	\vspace{0.5cm}
	\par
\newcolumntype{C}[1]{>{\centering\arraybackslash}p{#1}}
\newcolumntype{R}[1]{>{\raggedright\arraybackslash}p{#1}}
\renewcommand{\arraystretch}{1.2}
\vspace{-0.5cm}
\begin{flushleft}
	\begin{longtable}{|R{4cm}|R{2.2cm}|R{2.2cm}|C{2.2cm}|C{2.2cm}|C{2.2cm}|}
		\hline
		Course Title                      & \multicolumn{5}{l|}{ \textbf{AEROSPACE STRUCTURAL DYNAMICS}}                                                                  \\ \hline
		Course Code                       & \multicolumn{5}{l|}{AAEC35}                                                                              \\ \hline
		Program                           & \multicolumn{5}{l|}{B.Tech}                                                                              \\ \hline
		Semester                          & VII                         & \multicolumn{4}{l|}{AE}                                                   \\ \hline
		Course Type                       & Core              & \multicolumn{4}{l|}{}                                                     \\ \hline
		Regulation                        &      UG-20         & \multicolumn{4}{l|}{.}                                                     \\ \hline
		\multirow{3}{*}{Course Structure} & \multicolumn{3}{c|}{Theory}                                             & \multicolumn{2}{c|}{Practical} \\ \cline{2-6} 
		& \multicolumn{1}{c|}{Lecture} & \multicolumn{1}{c|}{Tutorials} & Credits & Laboratory      & Credits      \\ \cline{2-6} 
		& \multicolumn{1}{c|}{3}       &       1    & 4       & -               & -            \\ \hline
		Course Coordinator                 & \multicolumn{5}{l|}{Mr. K Arun Kumar, Assistant Professor}                                          \\ \hline                                      
	\end{longtable}
\end{flushleft}
\vspace{-1cm}
%\textcolor{blue}{\large \bfseries COURSE OBJECTIVES:}
	\flushleft\textbf{\textcolor{blue}{\large COURSE OBJECTIVES:}}\\		
\textbf{The students will try to learn:}\vspace{-0.5cm}
\newcolumntype{C}[1]{>{\centering\arraybackslash}p{#1}}
\newcolumntype{R}[1]{>{\raggedright\arraybackslash}p{#1}}
\renewcommand{\arraystretch}{1.5}
\begin{flushleft}	
\begin{longtable}{|C{1.5cm}|R{15cm}|}
	\hline
		I & Formulate mathematical models of problems in vibrations using Newton’s second law or energy principles.\tabularnewline
	\hline
	II &Determine a complete solution to the modelled mechanical vibration problems.\tabularnewline
	\hline
	III &    design a mechanical system that has desirable vibrational behavior.\tabularnewline
	\hline	
	IV & Assess the underlying assumptions in the aeroelastic analysis of fixed wing and rotary
	wing aerospace vehicles/systems.\tabularnewline
	\hline
\end{longtable}
\end{flushleft}
\vspace{-1cm}
\flushleft\textbf{\textcolor{blue}{\large COURSE OUTCOMES:}}\\
%\textcolor{blue}{\large \bfseries COURSE }\\
\textbf{After successful completion of the course, students should be able to:}
\renewcommand{\arraystretch}{1.1}
\vspace{-0.5cm}
\begin{flushleft}
\begin{longtable}{|C{1.2cm}|R{13cm}|C{2cm}|}
	\hline
	CO 1 &	\textbf{Outline} \textcolor{blue}{ the fundamental concepts of mechanical vibrations} \textcolor{red}{and justify their application in a variety of engineering design contexts}&Apply\tabularnewline
\hline
CO 1&	\textbf{Analyze} \textcolor{blue}{  Analytically and numerically predict the dynamic response of a single degree-of-freedom mass-spring-damper system } \textcolor{red}{with no force excitation, with harmonic force excitation, and with general force excitation.}&	Analyze\tabularnewline
\hline
CO 3&	\textbf{Compute} \textcolor{blue}{  the natural frequency (or
	frequencies) of vibratory systems  }\textcolor{red}{ for determining the system's modal response.}&	Apply\tabularnewline
\hline 

CO 4&	\textbf{Apply} \textcolor{blue}{  theoretical and numerical procedures } \textcolor{red}{for predicting the dynamic response of  continuous structural systems under the most diverse loading conditions}.	&Apply\tabularnewline
\hline		
CO 5&\textbf{Formulate } \textcolor{blue}{ the static aeroelasticity problems such as typical section and wing divergence
	problems; }\textcolor{red}{ for their selection in real world
	applications.}	&	Apply\tabularnewline\hline
\end{longtable}
\end{flushleft}
\vspace{-1cm}
\flushleft\textbf{\textcolor{blue}{\large DEFINITION AND TERMINOLOGY:}}\\
\vspace{-0.5cm}
	\begin{flushleft}
		\begin{longtable}{|>{\centering\arraybackslash}p{1.4cm}  |  >{\raggedright\arraybackslash}p{13cm} |>{\centering\arraybackslash}p{1.6cm}|} 
			\hline
			\textbf{S.No}&	\centering \textbf{DEFINITION}&		\textbf{CO's} \\
			\hline 
			\rowcolor{blue!35}\multicolumn{3}{| c |}{\textbf{MODULE I}}\\
			\hline 
			\rowcolor{yellow!35}\multicolumn{3}{| c |}{\textbf{SINGLE-DEGREE-OF-FREEDOM LINEAR SYSTEMS}}\\
			\hline 	
	1	& \textcolor{ForestGreen}{\textbf{Define Amplitude.}} & \multirow{2}{*}{CO 1} \\\cline{2-2}
		&The maximum displacement of a vibrating body from its equilibrium position is called the amplitude of vibration. &\\
		\hline	
	2&	\textcolor{ForestGreen}{\textbf{What is mean by theory of vibration?}} & \multirow{2}{*}{CO 1} \\\cline{2-2}
	&	Vibration is the motion of a particle or a body or a system of concentrated bodies
	having been displaced form a position of equilibrium, appearing as an oscillation. &\\	\hline
	3&	\textcolor{ForestGreen}{\textbf{What do you mean by Dynamic Response?}} & \multirow{2}{*}{CO 1} \\\cline{2-2}
	&	he Dynamic may be defined simply as time varying. Dynamic load is therefore
	any load which varies in its magnitude, direction or both, with time. The structural
	response (i.e., resulting displacements and stresses) to a dynamic load is also time
	varying or dynamic in nature. Hence it is called dynamic response &\\	\hline
	4&	\textcolor{ForestGreen}{\textbf{Define damping}} & \multirow{2}{*}{CO 1} \\\cline{2-2}
	&	Damping is a measure of energy dissipation in a vibrating system. The dissipating
	mechanism may be of the frictional form or viscous form. In the former case, it is called
	dry friction or column damping and in the latter case it is called viscous damping.
	Damping in a structural system generally assumed to be of viscous type for mathematical
	convenience. &\\	\hline
	5&	\textcolor{ForestGreen}{\textbf{Define Frequency.}} & \multirow{2}{*}{CO 1} \\\cline{2-2}
	&	Denoting how often something occurs, the same thing applies in vibration too. This denotes how frequently something occurs. For example, made to appear at regular intervals based on their relative motion.&\\	\hline
	6&	\textcolor{ForestGreen}{\textbf{Define Resonance}}& \multirow{2}{*}{CO 1} \\\cline{2-2}
	&	Resonance describes the phenomenon of increased amplitude that occurs when the frequency of an applied periodic force (or a Fourier component of it) is equal or close to a natural frequency of the system on which it acts.&\\	\hline
	7&	\textcolor{ForestGreen}{\textbf{What is meant by Degrees of freedom?}} & \multirow{2}{*}{CO 1} \\\cline{2-2}
	&	The minimum number of independent coordinates required to
	determine completely the position of all parts of a system at any instant of time defines the
	degree of freedom of the system. &\\	\hline\newpage\hline
	8&	\textcolor{ForestGreen}{\textbf{Define Period of oscillation.}} & \multirow{2}{*}{CO 1} \\\cline{2-2}
	&	The time taken to complete one cycle of motion is known as the period of oscillation $\tau =2 \phi/ \omega$
Time period and is denoted by $\tau$
Rotate through an angle of 2 $\Pi$
The circular frequency $\omega$.&\\	\hline
	9&	\textcolor{ForestGreen}{\textbf{Define Cycle.}} & \multirow{2}{*}{CO 1} \\\cline{2-2}
	&	The movement of a vibrating body from its undisturbed or equilibrium position to its extreme position in one direction, then to the equilibrium position, then to its extreme position in the other direction, and back to equilibrium position is called a cycle of vibration.&\\	\hline
	10&	\textcolor{ForestGreen}{\textbf{Define Frequency of oscillation.}} & \multirow{2}{*}{CO 1} \\\cline{2-2}
	&	The number of cycles per unit time is called the frequency of oscillation &\\	\hline
	11&	\textcolor{ForestGreen}{\textbf{Write a short note on simple Harmonic motion}} & \multirow{2}{*}{CO 1} \\\cline{2-2}
	&	Simple Harmonic Motion or SHM is defined as a motion in which the restoring force is directly proportional to the displacement of the body from its mean position. The direction of this restoring force is always towards the mean position.
 &\\\hline
	12&	\textcolor{ForestGreen}{\textbf{What is a Phase angle.}} & \multirow{2}{*}{CO 1} \\\cline{2-2}
	&	When a body does not seem to be acting in accordance with inertia, it is in a non-inertial frame of reference or accelerating. &\\\hline
	13&	\textcolor{ForestGreen}{\textbf{What is  Natural frequency.}} & \multirow{2}{*}{CO 1} \\\cline{2-2}
	&	The fictitious force causing the apparent deflection of moving objects when viewed in a rotating frame of reference. &\\\hline
	14&	\textcolor{ForestGreen}{\textbf{Define Octave.}} & \multirow{2}{*}{CO 1} \\\cline{2-2}
	&	It is the point along the rocket z axis with the same amount of surface area on both sides. &\\\hline
	15&	\textcolor{ForestGreen}{\textbf{Define Decibel.}} & \multirow{2}{*}{CO 1} \\\cline{2-2}
	&	The various quantities encountered in the field of vibration and sound (such as displacement, velocity, acceleration, pressure, and power) are often represented using the notation of decibel. &\\\hline
		16&	\textcolor{ForestGreen}{\textbf{Write a short note on linear time-invariant system.}} & \multirow{2}{*}{CO 1} \\\cline{2-2}
	&A linear time-invariant system describes the relationship
	between an input signal and an output signal. For example, the input
	signal could be a velocity v(t), and the output signal a force F(t) &\\\hline
	16&	\textcolor{ForestGreen}{\textbf{List classification of vibrations}} & \multirow{2}{*}{CO 1} \\\cline{2-2}
		&Classification of Vibration:
		 Free and forced
		 Damped and undamped
		 Linear and nonlinear
		 Deterministic and Random  &\\\hline
	\rowcolor{blue!35}\multicolumn{3}{| c |}{\textbf{MODULE II}}\\
	\hline
	
	\rowcolor{yellow!35}\multicolumn{3}{| c |}{ \textbf{TWO-DEGREE-OF-FREEDOM SYSTEMS
}}\\
	\hline 
		1&	\textcolor{ForestGreen}{\textbf{Define Resonance.}} & \multirow{2}{*}{CO 2} \\\cline{2-2}
		&	Whenever the natural frequency of vibration of a machine or structure coincides with the frequency of the external excitation, there occurs a phenomenon known as resonance.&\\\hline\newpage\hline
		2&	\textcolor{ForestGreen}{\textbf{Define vibration.}} & \multirow{2}{*}{CO 2} \\\cline{2-2}
		&	Any motion that repeats itself after an interval of time is called vibration or oscillation. &\\\hline
		3&	\textcolor{ForestGreen}{\textbf{Define generalized coordinates.}} & \multirow{2}{*}{CO 2} \\\cline{2-2}
		&	The coordinates necessary to describe the motion of a system constitute a set of generalized coordinates. These are usually denoted as and may represent Cartesian and/or non-Cartesian coordinates.&\\\hline
		4&	\textcolor{ForestGreen}{\textbf{Define discrete or lumped parameter systems}}& \multirow{2}{*}{CO 2} \\\cline{2-2}		&	Systems with a finite number of degrees of freedom are called discrete or lumped parameter systems&\\\hline
		5&	\textcolor{ForestGreen}{\textbf{Define continuous or distributed systems}} & \multirow{2}{*}{CO 2} \\\cline{2-2}		&	Systems with a finite number of degrees of freedom are called discrete or lumped parameter systems, and those with an infinite number of degrees of freedom are called continuous or distributed systems. &\\\hline
		6&	\textcolor{ForestGreen}{\textbf{Define Free Vibration.}} & \multirow{2}{*}{CO 2} \\\cline{2-2}
		&	If a system, after an initial disturbance, is left to vibrate on its own, the ensuing vibration is known as free vibration. No external force acts on the system. The oscillation of a simple pendulum is an example of free vibration.&\\\hline
		7&	\textcolor{ForestGreen}{\textbf{Define Forced Vibration.}} & \multirow{2}{*}{CO 2} \\\cline{2-2}
		&	If a system is subjected to an external force (often, a repeating type of force), the resulting vibration is known as forced vibration.&\\\hline
		8&	\textcolor{ForestGreen}{\textbf{When resonance will occur.}} & \multirow{2}{*}{CO 2} \\\cline{2-2}
		&	If the frequency of the external force coincides with one of the natural frequencies of the system, a condition known as resonance occurs.&\\
		9&	\textcolor{ForestGreen}{\textbf{Define undamped vibration.}} & \multirow{2}{*}{CO 2} \\\cline{2-2}
		&	If no energy is lost or dissipated in friction or other resistance during oscillation, the vibration is known as undamped vibration. &\\\hline
		10&	\textcolor{ForestGreen}{\textbf{Define linear vibration.}}& \multirow{2}{*}{CO 2} \\\cline{2-2}
		&	If all the basic components of a vibratory system the spring, the mass, and the damper behave linearly, the resulting vibration is known as linear vibration. &\\\hline
		11&	\textcolor{ForestGreen}{\textbf{Define nonlinear vibration.}} & \multirow{2}{*}{CO 2} \\\cline{2-2}
		&	If, however, any of the basic components behave nonlinearly, the vibration is called nonlinear vibration.&\\\hline
		12&	\textcolor{ForestGreen}{\textbf{Define deterministic. }}& \multirow{2}{*}{CO 2} \\\cline{2-2}
		&	If the value or magnitude of the excitation (force or motion) acting on a vibratory system is known at any given time, the excitation is called deterministic. &\\\hline
		13&	\textcolor{ForestGreen}{\textbf{Define deterministic vibration.}} & \multirow{2}{*}{CO 2} \\\cline{2-2}
		&	If the value or magnitude of the excitation (force or motion) acting on a vibratory system is known at any given time, the excitation is called deterministic. The resulting vibration is known as deterministic vibration. &\\\hline\newpage\hline
		14&	\textcolor{ForestGreen}{\textbf{Define Random vibration.}}& \multirow{2}{*}{CO 2} \\\cline{2-2}
		&	If the excitation is random, the resulting vibration is called random vibration. &\\\hline
		15&	\textcolor{ForestGreen}{\textbf{Define Damped vibration.}}& \multirow{2}{*}{CO 2} \\\cline{2-2}
		&	If any energy is lost in this way, however, it is called damped vibration.&\\\hline
			\rowcolor{blue!35}\multicolumn{3}{| c |}{\textbf{MODULE III}}\\
		\hline 
		\rowcolor{yellow!35}\multicolumn{3}{| c |}{\textbf{MULTI-DEGREE-OF-FREEDOM LINEAR SYSTEMS}}\\
		\hline  
		1&	\textcolor{ForestGreen}{\textbf{Define Spring constant or spring stiffness or spring rate.}} & \multirow{2}{*}{CO 3} \\\cline{2-2}
		&	A spring is said to be linear if the elongation or reduction in length x is related to the applied force F as $F = kx$
Where k is a constant, known as the spring constant or spring stiffness or spring rate. &\\\hline
		2&	\textcolor{ForestGreen}{\textbf{Define Damping.}} & \multirow{2}{*}{CO 3} \\\cline{2-2}
		&	The mechanism by which the vibrational energy is gradually converted into heat or sound is known as damping. &\\\hline
		3&\textcolor{ForestGreen}{\textbf{ Define	Viscous damping.}} & \multirow{2}{*}{CO 3} \\\cline{2-2}
		&	In viscous damping, the damping force is proportional to the velocity of the vibrating body. &\\\hline
		4&	\textcolor{ForestGreen}{\textbf{Define Coulomb or Dry-Friction Damping.}} & \multirow{2}{*}{CO 3} \\\cline{2-2}
		&	The damping force is constant in magnitude but opposite in direction to that of the motion of the vibrating body. It is caused by friction between rubbing surfaces that either are dry or have insufficient lubrication. &\\\hline
		5&	\textcolor{ForestGreen}{\textbf{Summarize Hysteretic Damping.}} & \multirow{2}{*}{CO 3} \\\cline{2-2}
		&	When a material is deformed, energy is absorbed and dissipated by the material. The effect is due to friction between the internal planes, which slip or slide as the deformations take place. &\\\hline
		6&	\textcolor{ForestGreen}{\textbf{Define Periodic motion.}} & \multirow{2}{*}{CO 3} \\\cline{2-2}
		&	Oscillatory motion may repeat itself regularly, as in the case of a simple pendulum, or it may display considerable irregularity, as in the case of ground motion during an earthquake. If the motion is repeated after equal intervals of time, it is called periodic motion. &\\\hline
		7&	\textcolor{ForestGreen}{\textbf{Define Harmonic motion.}}& \multirow{2}{*}{CO 3} \\\cline{2-2}
		&	The simplest type of periodic motion is harmonic motion.&\\\hline
		8&	\textcolor{ForestGreen}{\textbf{What is Simple harmonic motion?}} & \multirow{2}{*}{CO 3} \\\cline{2-2}
		&	It can be seen that the acceleration is directly proportional to the displacement. Such a vibration, with the acceleration proportional to the displacement and directed toward the mean position, is known as simple harmonic motion.&\\\hline
		9&	\textcolor{ForestGreen}{\textbf{Define Torsional vibration. }}& \multirow{2}{*}{CO 3} \\\cline{2-2}
		&	If a rigid body oscillates about a specific reference axis, the resulting motion is called torsional vibration. &\\\hline\newpage\hline
		10&	\textcolor{ForestGreen}{\textbf{Define Orthogonality.}} & \multirow{2}{*}{CO 3} \\\cline{2-2}
		&	As the number of degrees of freedom increases, the solution of the characteristic equation becomes more complex. The mode shapes exhibit a property known as orthogonality.&\\\hline
		11&	\textcolor{ForestGreen}{\textbf{Define Proportional damping.}} & \multirow{2}{*}{CO 3} \\\cline{2-2}
		&	The solution of forced-vibration problems associated with viscously damped systems can also be found conveniently by using a concept called proportional damping. &\\\hline
		12&	\textcolor{ForestGreen}{\textbf{Define lumped-parameter or lumped-mass or discrete-mass systems.}}& \multirow{2}{*}{CO 3} \\\cline{2-2}
		&	The lumped masses are assumed to be connected by massless elastic and damping members. Linear (or angular) coordinates are used to describe the motion of the lumped masses (or rigid bodies). Such models are called lumped-parameter or lumped-mass or discrete-mass systems. &\\\hline
		13&	\textcolor{ForestGreen}{\textbf{Define Finite element method. }}& \multirow{2}{*}{CO 3} \\\cline{2-2}
		&	Method of approximating a continuous system as a multi degree-of freedom system involves replacing the geometry of the system by a large number of small elements. By assuming a simple solution within each element, the principles of compatibility and equilibrium are used to find an approximate solution to the original system. This method, known as the finite element method. &\\\hline
		14&	\textcolor{ForestGreen}{\textbf{What is Influence coefficients?}} & \multirow{2}{*}{CO 3} \\\cline{2-2}
		&	The equations of motion of a multi degree-of-freedom system can also be written in terms of influence coefficients, which are extensively used in structural engineering. Basically, one set of influence coefficients can be associated with each of the matrices involved in the equations of motion. &\\\hline
		15&	\textcolor{ForestGreen}{\textbf{Define Flexibility influence coefficients.}} & \multirow{2}{*}{CO 3} \\\cline{2-2}
		&	The influence coefficients corresponding to the inverse stiffness matrix are called the flexibility influence coefficients. &\\\hline
	
			\rowcolor{blue!35}\multicolumn{3}{| c |}{\textbf{MODULE IV}}\\
		\hline 
		\rowcolor{yellow!35}\multicolumn{3}{| c |}{ \textbf{DYNAMICS OF CONTINUOUS ELASTIC BODIES}}\\\hline
		1&	\textcolor{ForestGreen}{\textbf{What is Node?}} & \multirow{2}{*}{CO 4} \\\cline{2-2}
		&	The points at which $w_n=0 $ for all times are called nodes. &\\\hline
		2&	\textcolor{ForestGreen}{\textbf{Define Euler-Bernoulli or thin beam theory}} & \multirow{2}{*}{CO 4} \\\cline{2-2}
		&	From the elementary theory of bending of beams. &\\\hline
		3&	\textcolor{ForestGreen}{\textbf{What is Thick beam theory or Timoshenko beam theory}} & \multirow{2}{*}{CO 4} \\\cline{2-2}
		&	If the cross-sectional dimensions are not small compared to the length of the beam, we need to consider the effects of rotary inertia and shear deformation. Is known as the thick beam theory or Timoshenko beam theory.  &\\\hline
		4&	\textcolor{ForestGreen}{\textbf{Define Timoshenko shear coefficient.}} & \multirow{2}{*}{CO 4} \\\cline{2-2}
		&	Where G denotes the modulus of rigidity of the material of the beam and k is a constant, also known as Timoshenko s shear coefficient. &\\\hline\newpage\hline
		5&	\textcolor{ForestGreen}{\textbf{Define Rayleigh-Ritz method}}& \multirow{2}{*}{CO 4} \\\cline{2-2}
		&	Based on Rayleigh s quotient, for finding the approximate fundamental frequencies of continuous systems is outlined. The extension of the method, known as the Rayleigh-Ritz method. &\\\hline
		6&	\textcolor{ForestGreen}{\textbf{Define Distributed or continuous systems.}} & \multirow{2}{*}{CO 4} \\\cline{2-2}
		&	A continuous system is also called a system of infinite degrees of freedom.&\\\hline
		7&	\textcolor{ForestGreen}{\textbf{Define System of infinite degrees of freedom.}} & \multirow{2}{*}{CO 4} \\\cline{2-2}
		&	It involves knowing your starting point, knowing the location of your target, and using Newtonian laws of classical mechanics to launch a trajectory&\\\hline
		8&	\textcolor{ForestGreen}{\textbf{Define Wave equation.}} & \multirow{2}{*}{CO 4} \\\cline{2-2}
		&	The Equation $c^2 \frac{\text{d}^2w}{\text{d}x^2}=\frac{\text{d}^2w}{\text{d}t^2}$   is also known as the wave equation. &\\\hline
		9&	\textcolor{ForestGreen}{\textbf{Define Frequency or characteristic equation.}}& \multirow{2}{*}{CO 4} \\\cline{2-2}
		&	Equation $ \sin \omega t/c =0$   is called the frequency or characteristic equation. &\\\hline
		10&	\textcolor{ForestGreen}{\textbf{Define Eigen values.}}& \multirow{2}{*}{CO 4} \\\cline{2-2}
		&	Equation $\sin \omega t/c =0$ is called the frequency or characteristic equation and is satisfied by several values of $\omega$ The values of $\omega$ are called the eigen values (or natural frequencies or characteristic values) of the problem.&\\\hline
		11&	\textcolor{ForestGreen}{\textbf{Define Fundamental mode.}} & \multirow{2}{*}{CO 4} \\\cline{2-2}
		&	The mode corresponding to n = 1 is called the fundamental mode. &\\\hline
		12&\textcolor{ForestGreen}{\textbf{ Define Fundamental frequency.}} & \multirow{2}{*}{CO 4} \\\cline{2-2}
		&	The mode corresponding to n = 1 is called the fundamental mode, and $\omega_1$ is called the fundamental frequency.&\\\hline
		13&	\textcolor{ForestGreen}{\textbf{What is Torsional stiffness ?}} & \multirow{2}{*}{CO 4} \\\cline{2-2}
		&	Where G is the shear modulus and GJ(x) is the torsional stiffness. &\\\hline
		\rowcolor{blue!35}\multicolumn{3}{| c |}{\textbf{MODULE V}}\\
		\hline 
	\rowcolor{yellow!35}\multicolumn{3}{| c |}{\textbf{ INTRODUCTION TO AEROELASTICITY}}\\
		\hline 
		1&	\textcolor{ForestGreen}{\textbf{Define Aeroelasticity.}}& \multirow{2}{*}{CO 5} \\\cline{2-2}
		&Aeroelasticity is defined as the
		interaction between the aerodynamic forces
		and the structural forces causing a
		deformation in the structure of the aerospace
		craft as well as in its control mechanism or in
		its propulsion systems. &\\\hline
		2&	\textcolor{ForestGreen}{\textbf{Give few real life non-aeronautical related examples where aeroelastic problems are
				prominent.
		}}& \multirow{2}{*}{CO 5} \\\cline{2-2}
		&	"Bridge vibrations, Galloping of ice-covered electrical transmission lines – singing of wires
		Tall buildings or skyscrapers subjected to wind loading, Tall chimneys subjected to wind loading, Nuclear fuel rod,  Submarine periscopes"
		&\\\hline
		3&	\textcolor{ForestGreen}{\textbf{Explain static aeroelasticity.
		}} & \multirow{2}{*}{CO 5} \\\cline{2-2}
		&	Static aeroelasticity considers the nonoscillatory effects of aerodynamic forces acting on the flexible aircraft structure. The flexible nature of the wing will influence the in-flight wing shape and hence the lift distribution in a steady manoeuvre
		&\\\hline\newpage\hline
		4&	\textcolor{ForestGreen}{\textbf{Explain dynamic aeroelasticity.
		 }}& \multirow{2}{*}{CO 5} \\\cline{2-2}
		&Dynamic aeroelasticity is concerned with the oscillatory effects of the aeroelastic interactions . This instability involves two or more modes of vibration and arises from the unfavourable coupling of aerodynamic, inertial and elastic forces
		 &\\\hline
		5&	\textcolor{ForestGreen}{\textbf{What is wing divergence?
		}} & \multirow{2}{*}{CO 5} \\\cline{2-2}
		&	"phenomenon that occurs when the moments due to aerodynamic forces overcome the restoring moments
		due to structural stiffness, so resulting in structural failure"
		 &\\\hline
		6&	\textcolor{ForestGreen}{\textbf{What is aileron control reversal?
		}} & \multirow{2}{*}{CO 5} \\\cline{2-2}
		&"A static aeroelastic instability occurring in wings with ailerons in flight, at a speed called
		the control reversal speed. At this speed the ailerons reverse their usual functionality (i.e.,
		the rolling direction associated with the given aileron moment is reversed."
		&\\\hline
		7&	\textcolor{ForestGreen}{\textbf{What is buffeting?
		}} & \multirow{2}{*}{CO 5} \\\cline{2-2}
		&Buffeting is a high-frequency instability. It is caused by airflow separation or shock wave oscillations. It is a random forced vibration. Generally, it affects the tail unit of the aircraft structure due to the air flow downstream of the wing.
		&\\\hline
		8&	\textcolor{ForestGreen}{\textbf{Define Torsional divergence phenomena.}} & \multirow{2}{*}{CO 5} \\\cline{2-2}
		&	A major factor in the predominance of the biplane design until the early 1930s when “stressed skin” metallic structural configurations were introduced to provide adequate torsional stiffness form on planes. &\\\hline
		9&	\textcolor{ForestGreen}{\textbf{Outline the concept of flutter}} & \multirow{2}{*}{CO 5} \\\cline{2-2}
		&	Flutter is a self-feeding and potentially
		destructive vibration where aerodynamic
		forces on an object couple with a structure's
		natural mode of vibration to produce rapid
		periodic motion. &\\\hline
		10&	\textcolor{ForestGreen}{\textbf{Summarize Collar triangle of forces.
		}} & \multirow{2}{*}{CO 5} \\\cline{2-2}
		&Collar has classified problems in Aeroelasticity by means of a triangle of three types of forces aerodynamic,elastic and inertia, represented by the symbols A, E and I are placed at the vertices of a triangle. Each aeroelastic phenomenon can be located on thediagram according to its relation to the three vertices.
		 &\\\hline
		 	11&	\textcolor{ForestGreen}{\textbf{Give few real life aeronautical related examples where aeroelastic problems are
		 			prominent."		 					 }} & \multirow{2}{*}{CO 5} \\\cline{2-2}
		 & Wing divergence, Aileron control reversal
		 Flutter and buffeting
		 
		 &\\\hline
		 	12&	\textcolor{ForestGreen}{\textbf{What is dynamic response of an aircraft?
		 		 }} & \multirow{2}{*}{CO 5} \\\cline{2-2}
		 & Transient response of aircraft structural components  produced by suddenly applied loads due to gusts, landing,gun reactions, abrupt control motions, moving shock waves,or other dynamic loads.
		  &\\\hline
			\end{longtable}
	\end{flushleft}
	\begin{flushleft} \vspace{-0.3cm}
		\textbf{Course Coordinator: \hspace{10cm}\textbf{HOD, AE}\\ 
			Mr. K Arun Kumar,Assistant Professor  } \\
	\end{flushleft}
\end{document}